\chapter{Array's}
\label{cha:arrays}
\thispagestyle{empty}
\index{array}

Een \textsl{array} is een manier om bij elkaar behorende gegevens te groeperen en te bewerken.

Een array bestaat uit een aantal \textsl{elementen}. De nummering van de elementen begint bij 0.

De declaratie van een array begint met het opgeven van het datatype zoals \texttt{int} en \texttt{double}, gevolgd door de naam van de array en het aantal elementen. In onderstaande listing wordt de array \texttt{lijst} met tien elementen van het type \texttt{int} gedeclareerd.

\begin{lstlisting}[caption=Declaratie van een array met tien elementen]
int lijst[10];
\end{lstlisting}



\begin{figure}[!ht]
\centering
\begin{tikzpicture}[pointerstyle]
\foreach \ii [count=\i from 0] in {2,5,3,9,8,2,6,-1,10,0} {
	\node[memlocarray] (nod\i) at (\i*\unitsize,0) {};
%    \draw (nod\i.center) node {\ii};
	\draw (nod\i.center) node [yshift=\unitsize cm] {\i};
}
\node at (-0.6,0) [left] {lijst};
\end{tikzpicture}
\caption{Uitbeelding van een array met tien elementen.}
\label{fig:arryinmem}
\end{figure}
