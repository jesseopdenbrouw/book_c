\chapter{De ASCII-tabel}
\label{cha:asciitabel}
\thispagestyle{empty}

We hebben gezien dat binaire coderingen worden gebruikt om
numerieke gegevens (of informatie) weer te geven. Maar niet alle gegevens zijn numeriek.
Informatie kan ook bestaan uit letters en leestekens. Om er voor te zorgen dat
computers informatie kunnen uitwisselen is de ASCII-code bedacht.

\begin{table}[!p]
\caption{De ASCII-code~\cite{eijkhout2010ascii}.}
\label{tab:talascii-code}

\resizebox{\textwidth}{!}{\input{ascii.tex}}

%\bigskip\bigskip
%{\centering\small Deze tabel kan gevonden worden op \url{http://www.ctan.org/tex-archive/info/ascii-chart}}
\end{table}

De naam ASCII betekent \textsl{American Standard Code for Information Interchange} en
dat geeft al goed aan waarvoor de code bedoeld is: op een gestandaardiseerde wijze
informatie uitwisselen. De code is in 1963 voor het eerst gepubliceerd~\cite{asa1963ascii}
en in die tijd
was er nog geen noodzaak om andere tekens te gebruiken dan de bekende westerse letters,
cijfers en leestekens. Vandaar dat het aantal tekens beperkt is.

De ASCII-code bestaat uit 128 7-bits tekens zoals te zien is in
tabel~\ref{tab:talascii-code}. De tekens zijn verdeeld in leesbare tekens, zoals letters,
cijfers en leestekens en zogenoemde \textsl{besturingstekens}. De besturingstekens zijn nodig
\index{besturingsteken}
om informatie-overdracht af te bakenen en om een bepaalde \textsl{handshake} 
(uitwisselingsprotocol) te regelen.
Zo zijn er codes voor de \textsl{carriage return} (CR, code~$13_{10}$) en de
\textsl{backspace} (BS, code $8_{10}$). De eerste 32 tekens zijn besturingstekens
waarvan de meeste tegenwoordig niet meer gebruikt worden~\cite{maini2007digital}.

De codes zijn niet willekeurig toegekend dat we goed kunnen zien bij de cijfers
en letters. De tabel is zo opgesteld dat de cijfers elkaar opvolgen. Dat is
handig bij het afdrukken van een (decimaal) getal.
Hetzelfde geldt voor de letters, ook die volgen elkaar op. De makers hebben ook nagedacht
over de positie van hoofd- en kleine letters. Deze verschillen in de tabel in slechts
\'{e}\'{e}n bit (bit b$_{6}$ in de tabel). Bij het gebruik van de Caps Lock-toets
of Shift-toets hoeft dus maar \'{e}\'{e}n bit (in combinatie met een letter) gewijzigd
te worden.

Een aantal besturingstekens is ook in C direct te gebruiken.
Een besturingsteken begint altijd met een \textsl{backslash} (`\textbackslash')
gevolgd door een letter, cijfer of teken. Zo is het teken voor een horizontale tab
`\texttt{\textbackslash t}' en voor de \textsl{line feed} is het teken
`\texttt{\textbackslash n}'. Met een line feed gaat de cursor naar het begin van de
volgende regel.

