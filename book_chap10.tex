\chapter{De preprocessor}
\label{cha:preprocessor}
\thispagestyle{empty}

De C-compiler is uitgerust met een \textsl{preprocessor}\index{preprocessor}. Dit kan een apart programma zijn dat door de C-compiler zelf gestart wordt, of het kan ingebed zijn in de C-compiler.

De preprocessor is \textsl{geen} compiler in de zin van het genereren van uitvoerbare code voor een computer. De preprocessor verwerkt het C-bestand aan de hand van bepaalde opdrachten die allemaal beginnen met een \textsl{hash}\footnote{Met een woord erachter wordt dit een hashtag genoemd.} (\texttt{\#}). Die worden \textsl{directives} genoemd. De uitvoer van de preprocessor is een bestand waarin geen directive meer voorkomt en dat bestand wordt aan de C-compiler doorgegeven. De C-compiler heeft dus geen weet van directives.

\section{Invoegen van bestanden}
\index{include@{\texttt{\#include}}, preprocessor directive}
Een bestand kan in een C-programma worden ingevoegd door de \textsl{include directive}. Deze heeft de vorm:

\hspace*{1em}\texttt{\#include <}\textsl{bestandsnaam}\texttt{>}

of

\hspace*{1em}\texttt{\#include "{}}\textsl{bestandsnaam}\texttt{"}

Invoegen gebeurt op het punt waar \texttt{\#include} voorkomt. De include directive zelf wordt verwijderd. Een bestandsnaam kan tussen \texttt{<} en~\texttt{>} staan. Dan wordt gezocht in zogenoemde \textsl{systeemmappen}. Welke dat zijn, hangt van de toolchain af. Een bestandsnaam kan ook tussen aanhalingstekens staan. Dan wordt eerst in de map gezocht waarin ook het C-programma geplaatst is en, als het bestand niet gevonden is, in de systeemmappen. De include directive wordt typisch gebruikt om een \textsl{header-bestand}\index{header-bestand} aan het begin van een C-programma in te lezen. We hebben er al een aantal gebruikt:

\hspace*{1em}\texttt{\#include <stdio.h>}\\
\hspace*{1em}\texttt{\#include <stdint.h>}\\
\hspace*{1em}\texttt{\#include <string.h>}\\
\hspace*{1em}\texttt{\#include <math.h>}

De extensie \texttt{.h} geeft aan dat het om een header-bestand gaat. Dit is niet verplicht maar wel \textsl{good practice}. Het is \textsl{niet} de bedoeling om andere C-bestanden via de include directive in te lezen. Dat kan wel, maar is \textsl{bad practice}.

Header-bestanden kunnen zelf ook include directives bevatten, wat ook vaak het geval is. Daarbij moet erop gelet  worden dat header-bestanden niet meerdere keren geladen worden. Hoe dat vermeden kan worden, zien we in paragraaf~\ref{sec:conditioneelinlezen}.


\section{Macrovervanging}
\index{define@{\texttt{\#define}}, preprocessor directive}
Een \textsl{macro}\index{macro} in de eenvoudigste vorm, ziet eruit als:

\hspace*{1em}\texttt{\#define }\textsl{macronaam vervangende-tekst}

Overal waar \textsl{macronaam} voorkomt wordt dit vervangen door \textsl{vervangende-tekst} behalve in C-strings. De \textsl{macronaam} moet exact overeenkomen, dus als \texttt{MAX} gedefinieerd is, dan wordt er geen vervanging gedaan in \texttt{printf("{}MAX")} en \texttt{MAX\_GETAL}. Macro's mogen zelf ook weer macro's bevatten en het vervangen gaat net zolang door totdat er geen vervangingen meer worden gedaan. We noemen dit \textsl{macro-expansie}\index{macro-expansie}.

Een macro kan argumenten bevatten zodat de vervangende tekst voor verschillende doeleinden gebruikt kan worden. Een klassiek voorbeeld is:

\hspace*{1em}\texttt{\#define max(A, B) ((A) > (B) ? (A) : (B))}

De macro \texttt{max(A, B)} ziet eruit als een functie maar dat is het niet. Overal waar \texttt{max} voorkomt wordt het vervangen door de vervangende tekst. Merk op het gebruik van de vele haakjes. Dat is nodig om de argumenten correct uit te rekenen:

\hspace*{1em}\texttt{x = max(p+q, r+s);}

wordt dan vervangen door

\hspace*{1em}\texttt{x = ((p+q) > (r+s) ? (p+q) : (r+s));}

Een veelgemaakte fout is het als volgt definiëren van een kwadraat:

\hspace*{1em}\texttt{\#define sqr(x) x * x \ \ \ \ /* WRONG */}

en dat geeft een verkeerd antwoord als \texttt{x} wordt vervangen door \texttt{y+1}. De correcte manier is om extra haakjes te gebruiken:

\hspace*{1em}\texttt{\#define sqr(x) ((x) * (x)) \ \ \ \ /* correct */}

Het kan alsnog verkeerd gaan als we de increment- of decrement-operator gebruiken:

\hspace*{1em}\texttt{j = sqr(++i);}

Dit wordt geëxpandeerd tot \lstc{j = ((++i)} \lstc{* (++i))} dus i wordt twee keer verhoogd.
Een voordeel van een macro is dat de argumenten onafhankelijk zijn van het datatype. Als we een floating point-getal willen kwadrateren, dan kan dat ook:

\hspace*{1em}\texttt{double j = sqr(2.16);}

We kunnen de preprocessor ook een macro laten vergeten:

\index{undef@{\texttt{\#undef}}, preprocessor directive}
\hspace*{1em}\texttt{\#undef }\textsl{macronaam}


\section{Conditionele compilatie}
Het gebruik van macro's is bijzonder handig als we delen van een C-bestand wel of niet willen laten compileren. Een veelgebruikt voorbeeld is het afdrukken van debug-informatie:

\begin{lstlisting}[caption=Conditionele compilatie.]
#include <stdio.h>

#define DEBUG 1

int main(void) {
    // ...
#if defined(DEBUG)
    printf("Variabele: %d\n", a);
#endif
    // ...
}
\end{lstlisting}
\index{if@{\texttt{\#if}}, preprocessor directive}
\index{endif@{\texttt{\#endif}}, preprocessor directive}
\index{defined@{\texttt{defined}}, preprocessor directive}

In plaats van \texttt{defined} mogen we ook \texttt{\#ifdef} gebruiken. We kunnen ook header-bestanden aan de hand van macro's laten invoegen. De diverse C-compilers definiëren bepaalde macro's zodat deze in een C-bestand te gebruiken zijn, de zogenoemde \textsl{predefined macros}\index{predefined macros}. Een voorbeeld is te zien in listing~\ref{cod:precondinc}.

\begin{lstlisting}[caption=Conditioneel inlezen van header-bestanden.,label=cod:precondinc]
#include <stdio.h>

#ifdef __GNUC__                   /* GNU C-compiler */
#define VSTRING "GNU C-compiler"
#include <gnuc.h>
#endif
#ifdef _MSC_VER                   /* Microsoft C-compiler
#define VSTRING "Visual Studio C-compiler"
#include <ms.h>
#endif
#ifdef __clang__                  /* Clang C-compiler
#define VSTRING "Clang C-compiler"
#include <clang.h>
#endif

int main(void) {

	printf("Gecompileerd met " VSTRING);
	return 0;
}

\end{lstlisting}
\index{ifdef@{\texttt{\#ifdef}}, preprocessor directive}
\index{endif@{\texttt{\#endif}}, preprocessor directive}

Welke predefined macros gedefinieerd zijn, is afhankelijk van de C-compiler. De documentatie moet hiervoor geraadpleegd worden. Sommige macro's hebben ook nuttige waarden zoals te zien is in listing~\ref{cod:premscver}.

\begin{lstlisting}[caption=Gebruik van de \texttt{\_MSC\_VER}-macro.,label=cod:premscver]
#ifdef _MSC_VER
#if _MSC_VER == 1926
#define VSTRING "Visual Studio 2019 versie 16.6"
#elif _MSC_VER == 1925
#define VSTRING "Visual Studio 2019 versie 16.5"
#elif _MSC_VER == 1924
#define VSTRING "Visual Studio 2019 versie 16.4"
#else
#define VSTRING "Visual Studio 2019 versie 16.3 of ouder"
#endif
#endif
\end{lstlisting}
\index{elif@{\texttt{\#elif}}, preprocessor directive}
\index{else@{\texttt{\#else}}, preprocessor directive}

Meerdere macro's mogen in een logische expressie voorkomen. De onderstaande regels zorgen ervoor dat er conditioneel gecompileerd wordt als de macro \texttt{\_\_linux\_\_} wel bestaat en de macro \texttt{\_\_ANDROID\_\_} niet bestaat.

\hspace*{1em}\texttt{\#if defined(\_\_linux\_\_) \&\& !defined(\_\_ANDROID\_\_)}

In principe mag de expressie willekeurig complex zijn, als het maar resulteert in een constante integer. Typische C-constructies kunnen niet gebruikt worden, zoals (C-)variabelen, \texttt{sizeof} en expliciete type casts want de preprocessor is geen C-compiler.


\section{Conditioneel inlezen van een bestand}
\label{sec:conditioneelinlezen}
Om ervoor te zorgen dat tijdens een compilatie van een C-bestand een header-bestand niet meer dan één keer wordt ingelezen, kunnen we een macro gebruiken. Aan het begin van het header-bestand testen we of een bepaalde macro bestaat. Als dat \textsl{niet} zo is, dan definiëren we de macro en laden de rest van het header-bestand. De volgende keer dat het header-bestand (in dezelfde compilatieslag) wordt geladen is de macro gedefinieerd en wordt de code overgeslagen. Zie listing~\ref{cod:preheaderinc}.

\begin{lstlisting}[caption=Gebruik van een macro om een header-bestand één keer in te lezen.,label=cod:preheaderinc]
/* File myheader.h */

#ifndef _MYHEADER_H
#define _MYHEADER_H

    // ... contents goes here
    
#endif
\end{lstlisting}
\index{ifndef@{\texttt{\#ifndef}}, preprocessor directive}
